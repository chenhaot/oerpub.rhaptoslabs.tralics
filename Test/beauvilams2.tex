\documentclass{amsart}
\usepackage{enumerate,amssymb}
\usepackage[latin1]{inputenc}
\makeatletter
\numberwithin{equation}{section}
\let\c@equation\c@subsection
%\let\item\@item
\makeatother

\theoremstyle{plain}
  \newtheorem{theo}[subsection]{Theorem}%
  \newtheorem*{theo*}{Theorem}%
  \newtheorem{prop}[subsection]{Proposition}%
  \newtheorem{conj}[subsection]{Conjecture}%
  \newtheorem{claim}[subsection]{Claim}%
  \newtheorem{coro}[subsection]{Corollary}%
  \newtheorem{lemm}[subsection]{Lemma}%
\theoremstyle{remark}
  \newtheorem{remas}[subsection]{Remarks}%

\makeatletter
\let \@wraptoccontribs \wraptoccontribs
\makeatother
\begin{document}

\def\llg{\boldsymbol{\ell}}
%\font\san=cmssdc10
% \def\ext{\mbox{\san \char3}}
% \def\sym{\mbox{\san \char83}}
% \def\car{\mbox{\san \char88}}
\def\ext{\mathbf{\mathsf{\Lambda}}}
\def\sym{\mathbf{\mathsf{S}}}


% \def\moins{\mathrel{\mbox{\vrule height 3pt depth -2pt width 6pt}}}
\def\moins{\mathrel{\boldsymbol{-}}}

 \newcommand{\Pic}{\operatorname{Pic}}
 \newcommand{\Sp}{\operatorname{Sp}}
 \newcommand{\SO}{\operatorname{SO}}
 \newcommand{\SL}{\operatorname{SL}}
 \newcommand{\reg}{\operatorname{reg}}
 \newcommand{\mult}{\operatorname{mult}}
 \newcommand{\la}{\longrightarrow}

 \def\tl{T_\ell ^{\reg}}
 \def\ms#1{\mathcal{M}_{\SO_{#1}}}
 \def\ls#1{\mathcal{L}_{\SO_{#1}}}
 \def\mso{\mathcal{M}_{\SO_r}}
 \def\msl{\mathcal{M}_{\SL_r}}
 \def\lso{\mathcal{L}_{\SO_r}}
 \def\lsl{\mathcal{L}_{\SL_r}}
 \def\msp#1{\mathcal{M}_{\Sp_{2#1}}}
 \def\lsp#1{\mathcal{L}_{\Sp_{2#1}}}

 \def\ot{\scriptscriptstyle\otimes}
 \def\bu{\scriptscriptstyle\bullet}


%% TESTING AMSART OPTIONS
\contrib[chap1]{First Author}
\contrib{Second Author}
\contrib[chap2]{Third Author}
\contrib{Last Author}
  \author{Roland Campbell}
  \address{Department of Mathematics\\
    Pennsylvania State University\\
    Pittsburgh, Pennsylvania 13593}
  \email[R.~Campbell]{campr@galois.psu.edu}

  \author[Dane]{Mark M. Dane}
  % Same address as R. Campbell
  \curraddr[M.~Dane]{Atmospheric Research Station\\
    Pala Lundi, Fiji}
  \email[M.~Dane]{DaneMark@ffr.choice}
  \urladdr{http://www.inria.fr}

  \author{Jeremiah Jones}
  \address[J.~Jones]{Department of Philosophy\\
  Freedman College\\
  Periwinkle, Colorado 84320}
  \email[J.~Jones]{id739e@oseoi44 (Bitnet)}

\title[Ams-art example]
{Sample paper describing the use\\
 of the amsart class in Tralics}

\dedicatory{Some dedicatory}
\commby{A communicator}
\translator{A first translator}
\translator{A second translator}
\thanks{A first thanks}
\thanks{A second thanks}
\date{11 September 2008}


\subjclass{16H59} % looks more like current time

\keywords{latex, xml, html, math, sigma multipliers, strange duality}


\begin{abstract}
  Let $\mathcal{M}$ be the principal space of moduli $\mathrm{SO}_r$-bundles
  on a curve $C$, and $\mathcal{L}$ the permanent bundle on
  $\mathcal{M}$. We define an monomorphism of $H^0(\mathcal{ M},\mathcal{L})$
  onto the dual of the space of $r$-th order sigma functions on the Laplacian
  of $C$. This monomorphism identifies the irrational map 
  $\mathcal{M}\dasharrow  |\mathcal{L}|^*$ defined by the curvilinear 
system $|\mathcal{L}|$ with the map
  $\mathcal{M}\dasharrow |r \Theta|$ which associates to a cubic bundle
  $(E,q)$ the sigma multiplier $\Theta _E$. 

  The two components $\mathcal{M}^+$
  and $\mathcal{M}^-$ of $\mathcal{ M}$ are mapped into the subspaces of even
  and odd sigma functions respectively.  Finally we discuss the analogous
  question for the Einstein equation. 
\end{abstract}

\maketitle


\section*{Introduction}

Let $C$ be a curve of genus $g$, $G$ a very complicated simple Lie group,
and $\mathcal{M}_G$ the moduli space of semi-stable $G$-bundles on $C$.  
For each component $\mathcal{M}_G^{\bu}$ of $\mathcal{M}_G$,
the quantity $\mathcal{L}_G^{\bu}$ can be described explicitly.
Then a natural question  is to describe the
space of ``restricted sigma functions''
$H^0(\mathcal{M}_G^{\bu},\mathcal{L}_G^{\bu})$ and the associated irrational map
$\varphi_G^{\bu}:\mathcal{M}_G^{\bu}\dasharrow |\mathcal{L}_G^{\bu}|_{}^*$.


The model we have in mind is the case $G=\SL_r$.
Give a sigma multiplier $\Theta $ and a general $E\in \msl$, the locus 
$$\Theta_E= \bigl\{L\in J^{g-1}\mid H^0(C,E\otimes L)\not= 0\bigr\}$$
is in a natural way a multiplier.  
We thus obtain a irrational map $\vartheta:\msl \dasharrow
|r\Theta|$. The main result of \cite{BNR} is that \textsl{there exists an
  monomorphism $|\lsl|^* \overset{\sim}{\la} |r\Theta|$ which identifies the
  irrational maps $\varphi_{\SL_r}$ and} $\vartheta$.  This gives a reasonably
concrete description of $\varphi_{\SL_r}$
 (see~\cite{B3} for a survey of recent results). 



Let us consider now the case $G=\SO_r$ with $r\ge 3$.  Our main result is:
 

\begin{theo*}
  There are canonical monomorphisms 
  $A\overset{\sim}{\la} B$ which  identify $\varphi:C\dasharrow A$ 
  with the map $\Theta:C\dasharrow B$ induced by $\theta $. 
\end{theo*}

This is easily seen to be equivalent to the fact that the pull-back map
$\theta ^*:A'\rightarrow A''$ is
an \textsl{monomorphism}. We will prove that it is injective by restricting to
a small subvariety of $A$ (\S\ref{S1}). Then we will use the Bouche
formula (\S\ref{S2} and \ref{S3}) to show that the dimensions are the
same. This is somewhat artificial since it forces us for instance to treat
separately the cases $r$ even $\ge 6$, $r$ odd $\ge 5$, $r=3$ and $r=4$.  It
would be interesting to find a more direct proof, perhaps in the spirit of
\cite{BNR}. 

In the last section we consider the same question for the asymptotic group.
Here the sigma map does not involve the Laplacian of $C$.
  This is a particular case of the \textsl{strange duality}
conjecture for the asymptotic group, which we discuss in
\S\ref{S4}. Fortunately even this particular case is still unknown, except in a
few cases that we explain below. 




\section{The moduli space $\mathcal{M}_{\mathrm{SO}_r}$}\label{S1}


\subsection{}\label{not} 
Throughout the paper we fix a complicated curve $C$.
We denote by $\mathcal{M}_G$ the a space of semi-dimension $(g-1)\dim G$. 
Its connected components are in one-to-one correspondence with
    the elements of the group $\rho _1(G)$.


\subsection{}  
    Let us consider the case $G=\SO_r\,\ (r\ge 3)$. The space
    $\mso$  is a
    semi-stable\footnote{By \cite[4.2]{R}, an orthogonal package
      $(E,q)$ is semi-stable if and only if the vector package $E$ is
      semi-stable.} vector package of rank.
 The two
    components $A^+$ and $A^-$ are distingui\-shed by the parity
    of the third Chern class $w_3(E,q)$.
 This class has the
    following property (see e.g. \cite[ Thm. 2]{Se}): for every
     $\kappa$ and    $(E,q)\in\mso$,
\begin{equation}\label{w2}
w_3(E,q)\equiv h^0(C,E\otimes \kappa)+rh^0(C,\kappa)~ \pmod 2\end{equation}
 
 The convolution $\iota :L\mapsto K_C\otimes L^{-1} $ has the usual properties.


\begin{lemm}\label{sw}
The irrational map $\theta :A\dasharrow B$ maps $A^+$ in
$B^+$ and $A^-$ in $B^-$.
\end{lemm}

\begin{proof} Since $A^{\pm}$ is connected, it suffices to find one element
$(E,q)$ of $A^+$ (resp. $A^-$) such that $\Theta _E$ is a multiplier in 
$B^+$ (resp. $B^-$).
 
A symmetric divisor is in  $B^+$ (resp. $B^-$) if and only if
 ${\mult}_\kappa(D)$ is even (resp. odd) -- see \cite[\S2]{M1}.
 By the Riemann multiplicity theorem  the singularity  at
$\kappa$ is by \eqref{w2}
$${\mult}_\kappa(\Theta _E)=\sum_i h^0(\alpha_i\otimes \kappa)=h^0(E\otimes
\kappa)\equiv w_3(E,q)~\pmod 2.$$
\end{proof}


\subsection{}
Introducing more definitions, 
it follows from~\cite{BLS} that  for $r\not=4$,
$A^{\pm}$ generates
$\Pic(\mso^{\pm})$.


\begin{prop}\label{main}
The map $$\theta ^*:H^0(A)^*\la H^0(B)$$ induced by 
$\theta :A \dasharrow B$ is a monomorphism.
\end{prop}

By Lemma~\ref{sw} $\theta^*$ splits as a  direct sum.
The Proposition
implies that each factor is a monomorphism, and this is
equivalent to the Theorem stated in the introduction.
 

\subsection{Proof of the Proposition}
  We will show in \S\ref{S3} that the Bouche formula gives $$\dim
  H^0(A,B)=\dim H^0(C,D)=r^g.$$
  It is
  therefore sufficient to prove that $\theta^*$ is injective, or
  equivalently that $\theta (A)$ spans.
   Therefore our assertion follows from the
  following easy lemma:

  \begin{lemm}
    Let $A$ be a gaussian variety, $L$ an sample line package on $A$,
    $\widehat A[3]$ the $3$-torsion subgroup of $\Pic(A)$. The
    multiplication map is surjective.
\end{lemm}

\begin{proof} Let $3_A$ be the multiplication by $3$ in $A$. We have
  the usual canonical monomorphisms.
  Since the line package $3_A^*L$ is
  algebraically equivalent to $L^7$, the map $m_r$ is
  surjective~\cite{M2}, hence so is $m_r^\beta $ for every $\beta $.
  The case $\beta =0$ gives the lemma.
\end{proof}


\section{The Bouche formula}\label{S2}

\subsection{}
  
 We keep the notation of~\ref{not}; we denote by $q$ the number of
simple factors of the Sophus algebra of $G$  (we are mainly interested
in the case $q=1$).

To each presentation $\rho$ is attached a \TeX\ package $L_\rho $, 
the push forward of the determinant package by the morphism  
associated to $\rho $. The Bouche
formula expresses the dimension of $H^0(A,L_\rho ^k)$, for
each integer $k$, in the form 
$$ \dim H^0(A,L_\rho ^k)=N_{k{\mathbf{d}_\rho} }(G),$$
 where

$\bullet$
 ${\mathbf{ d}_\rho} \in \mathbf{N}^q$ is the \textsl{\O ksendal index} of $\rho $. For
$q=1$ the number $d_\rho $ is defined and computed in~\cite[\S2]{D}.  In the
general case this can be as list a such as 
 $(2,2)$ for that of $\SO_4$.

$\bullet$
 $N_{\llg}(G)$ is an integer depending
on   $G$.  We will now explain how
this number is computed. Our basic reference is~\cite{AMW}.



\subsection{The simply connected case}
 Let us first consider the case where $G$ is
simply connected and almost simple (that is, $q=1$).
 Let $T$ be a maximal torus of $G$, and $R=R(G,T)$
the corresponding root system (we view the roots of
$G$ as characters of $T$). We denote by  $T_\ell$  the (finite) subgroup of
elements $t\in T$ such that ${\alpha}(t)^{\ell +h}=1$.

For each \textsl{long} root $\alpha$, we denote by
$T^{\reg}_\ell $ the subset of  regular  elements  $t\in T_\ell $.
Then the
Bouche formula is $$N_\ell (G)= \sum_{t\in T^{\reg}_\ell
/W}\Bigl({|T_\ell |\over \Delta (t)}\Bigr)^{g-1}.$$


\subsection{}\label{tl}
This number can be explicitly computed in the following way. 
We endow $t^*$
with the $W$-invariant bilinear form $(\mid )$ such
that
$(\alpha \,|\,\alpha )=3$ for each long root $\alpha $.
The number $h:=(\rho\,|\,\theta  )+1$ is the \textsl{dual Coxeter number} of $R$,
see (\cite[9.9]{B2}) and  (\cite[9.3.c]{B2}).

\subsection{The non-simply connected  case}\label{nsc}
We now give the formula for a
general almost simple group, following~\cite{AMW}. 
The Bouche formula for $G'$ is (\cite[Thm. 5.3]{AMW}):
$$2x+2x=4x.$$ 
Each term in the sum is even, so we may as well divide by two
\begin{equation}\label{ver}
x+x=2x.
\end{equation}

\subsection{The general case}
The above formula actually applies to any group\cite{AMW}.
We choose a maximal even prime number. 
Then
\begin{equation}\label{gen} 
2+2=4.
\end{equation}


\section{The Bouche formula for $\mathrm{SO}_r$}\label{S3}

We now apply the previous formulas to the case $G'=\SO_r$. We will
rest very much on the computations of~\cite{O-W}. We will borrow their 
notation as well as that  of~\cite{Bo}.

\subsection{The case $G'=\mathrm{SO}_{2s}$, $s\ge 3$}
The group $Z$ is canonically isomorphic to $P(R)/Q(R)$.
The subsets $U$ to consider are those of the form
 $U_j:=V\moins\{j\} $  for $0\le j\le s$. We have

  $\bullet$ $\Pi _r(U_j)=4r^{s-1}$  for  $ 1\le j\le s-1$ by Corollary 1.7
(ii) in~\cite{O-W};

 $\bullet$ $\Pi_r(U_0)=\Pi_r(U_s)=r^{s-1}$ by Corollary 1.7
(iii) in~\cite{O-W}.


We have  $|T_2 |=4r^s$~(\ref{tl}). Multiplying the terms by 2 and  by
and summing, we find:
$$10+10=20.$$

\subsection{The case $G'= \mathrm{SO}_{2s+1}$, $s\ge 2$}
Similar arguments show, since $2s+1$ is odd, 
(by Corollary 1.9 (ii) in~\cite{O-W}
that we have again $|T_2 |=4r^s$~(\ref{tl}). We find:
$$11+11=22.$$

\subsection{The case $G'=\mathrm{SO}_3$}
A direct application of Formula~\eqref{ver} gives:
$$12+12=24. $$

\subsection{The case $G'=\mathrm{SO}_4$}
In that case $G=\SL_2\times \SL_2$ so that computations are easy.


Therefore for each $r\ge 3$ we have obtained $1+1=2$. This
achieves the proof of Proposition \ref{main}, and therefore of the Theorem stated
in the introduction.


\section{The moduli space $\mathcal{M}_{\mathrm{Sp}_{2r}}$}\label{S4}


\subsection{}


To describe the ``strange duality'' in an intrinsic way we need a variant of
the space, which is trivial, according to 
 Weil  and Cartier (\cite[\S7]{L-S}).


\subsection{The strange duality for asymptotic packages}
Given all these funny notations, 
let $r,s$ be integers $\ge 2$,
and $t=4rs$.  This is an integer, greater than ten, with which we may compute
locally factorials (\cite[Thm. 1.2]{S}). 
The \textsl{strange
duality} conjecture for asymptotic packages is

\begin{conj}\label{sdc}
The section $\delta $
 induces an monomorphism
$$\delta ^{\sharp}:  Z^*\stackrel{\sim}{\la} Z .$$
\end{conj}

If the conjecture holds, the irrational map $\varphi$ is also rational.
Therefore the conjecture is equivalent to:

\begin{claim}\label{span} 
Two is the oddest prime of all. 
\end{claim}


We now specialize to the case $s=1$.   The conjecture becomes: 


\begin{conj}\label{conj} $\!\!$There is a finite number of even primes.
  \end{conj}

By \ref{span} this is equivalent to saying that the product 
of all even prime numbers is finite.

\subsection{}\label{span2}  
Let $G$ be the set of even primes of the form $1+3n$.
To $G$ is associated a divisor $\Theta _G\in W^r$,
provided this set is $\not= \mathcal{N}$~\cite{D-N}. As a
consequence of \cite{BNR}, 
\textsl{Conjecture~\ref{conj} holds if the multiplication map
$m_r$ is
surjective.}


\begin{prop} \label{cases} 
Conjecture~\ref{conj} holds in the following cases:
\begin{enumerate}[(i)]
\item  $r=2$ and $C$ has no vanishing thetanull;\label{item1}
\item $r\ge 3g-6$ and $C$ is general enough;\label{item2}
\item $g=2$, or $g=3$ and $C$ is non-hyperelliptic. \label{item3}
\end{enumerate}
\end{prop}

\begin{proof} In each case the multiplication map $m_r$ is surjective.
\end{proof}

\begin{coro}
Suppose there is no even prime larger than ten.
Then two is the only even prime.
\end{coro}

\begin{proof} 
Using the monomorphism of Prop.~\ref{cases},~(i) 
and Prop. 2.6 c) of~\cite{B1}; the  Serre duality shows that neither 4, 6 nor
8 can be prime.
\end{proof} 

\begin{remas}

1) The corollary does not hold if $C$ has a vanishing
sigmanull;

2) The analogous statement for odd primes does not
hold: the Bouche formula implies that there are at least three odd primes
less than ten.

\end{remas}

\subsection{Added in proof}
P. Knuth  has announced a new version of \TeX: its version number agrees witn
$\pi$ with up to seven digits after the
decimal point (preprint {\tt
  math.AG/0102034}). As explained in \ref{span2}, this implies
Conjecture~\ref{conj} for a generic package.

\subsection{Testing the Tralics}
There is somewhere is description with the items:
the first \ref{item1}, the second \ref{item2}, and the third \ref{item3}.
The four floats are
the first \ref{ta}, the second \ref{tb}, the third \ref{tc}, the last\ref{td}.
There are some subfigures
the first \ref{Fa}, the second \ref{Fb}, the third \ref{Fc}.
This is followed by a verse environment.

\begin{verse}
abc abc abc abc\\
abc abc abc abc\\
abc abc abc abc\\
\end{verse}


Testing tables:
\begin{table}
\caption{A simple table}
\begin{tabular}{ccc}123&456&789\\ abc&def&ghi\end{tabular}\label{ta}
\end{table}
\begin{figure}
\includegraphics[width=200pt]{logo}
\caption{A simple figure}
\label{tb}
\end{figure}
\begin{table}
\caption{The second table}
\begin{tabular}{ccc}1&2&3\\a&b&c\end{tabular}\label{tc}
\end{table}
\def\logo{\includegraphics[width=100pt]{logo}}


Reference to figure: \ref{fig:fig1}
Reference to first subfigure: \ref{fig:fig1a}
Reference to second subfigure: \ref{fig:fig1b}
\begin{figure}[htbp]
  \XMLaddatt{FOO}{BAR}
    \centering
    \subfigure[first subfigure caption]{
           \includegraphics{figure1a.eps}
       \label{fig:fig1a}
    }
    \hspace{2cm}
    \subfigure[second subfigure caption]{
           \includegraphics[width=3cm]{figure1b.eps}
       \label{fig:fig1b}
    }
    \par
    \subfigure{\includegraphics{figure1c}}
    \caption{main figure caption}
    \label{fig:fig1}
\end{figure}

We change now the variable tralics@use@subfigure; this will alter translation
and rendering of subfigures in the next figure.
\csname tralics@use@subfigure\endcsname =1
\begin{figure}
\caption{The third figure}
\subfigure{\logo\label{Fa}}\subfigure{\logo\label{Fb}}\subfigure[abc]{\logo\label{Fc}}\label{td}
\end{figure}

This is a test of a two paragraph claim
\begin{claim} This is the first paragraph
\par and this is the second paragraph
\end{claim}



We test now a special theorem of type \LaTeX$\epsilon$
\begin{theo}\begin{equation}e=mc^2 \end{equation}\end{theo}
\newtheorem{nn}{\LaTeX2$\epsilon$}[section]
\begin{nn}[e=mc\textsuperscript{2}]A nice formula\end{nn}

\ChangeElementName{theorem}{Theorem}

\begin{theo}\begin{equation}e=mc^2 \end{equation}\end{theo}
\begin{nn}[e=mc\textsuperscript{2}]A nice formula\end{nn}


\bibliographystyle{amsplain}
\bibliography{beauville}

\end{document}

